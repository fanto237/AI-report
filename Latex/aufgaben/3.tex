\section{Aufgabestellung}
\begin{itemize}
    \item Lassen Sie 3 frei ausgewählte Zahlenbilder erkennen und schreiben
          Sie dazu die Wahrscheinlichkeiten auf. (MNIST-Datenbank).
    \item Probieren Sie die Anzahlen von (Tag * Monat von Ihrem
          Geburtstag) Knoten aus.
    \item Zeichen Sie folgende Diagramme:
          \begin{itemize}
              \item Die Abhängigkeit der Trefferquote von der Anzahl der Knoten in der
                    versteckten Schicht.
              \item Quotenverlauf über der Anzahl der Epochen für die Lernraten.
              \item Ergebnisse in Abhängigkeit von der Anzahl der Epochenahl der
                    Epochen für die Lernraten.
              \item Einfluss der Lernrate auf die Erfolgsquote.
          \end{itemize}
    \item Probieren Sie verschiedene Anzahlen von versteckten Knoten
          aus, ändern Sie die Skalierung der Daten und verwenden Sie
          eine andere Aktivierungsfunktion (Optional)
\end{itemize}
\pagebreak

\section{Lösungsverfahren}
